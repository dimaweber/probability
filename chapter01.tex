\chapter{Probability Models}\label{chap:probability_models}
\minitoc
This chapter introduces the basic concept of the entire course, namely, probability. We discuss why probability was introduced as a scientific concept and how it has been formalized mathematically in terms of a probability model. Following this we develop some of the basic mathematical results associated with the probability model.

\section{Probability: A Measure of Uncertainty}\label{master_of_uncertain}
Often in life we are confronted by our own ignorance. Whether we are pondering tonight's traffic jam, tomorrow's weather, next week's stock prices, an upcoming election, or where we left our hat, often we do not know an outcome with certainty. Instead, we are forced to guess, to estimate, to hedge our bets.

Probability is the science of uncertainty. It provides precise mathematical rules for understanding and analyzing our own ignorance. It does not tell us tomorrow's weather or next week's stock prices; rather, it gives us a framework for working with our limited knowledge and for making sensible decisions based on what we do and do not know.

To say there is a 40\% chance of rain tomorrow is not to know tomorrow's weather. Rather, it is to know what we do
not know about tomorrow's weather.

In this text, we will develop a more precise understanding of what it means to say there is a 40\% chance of rain
tomorrow. We will learn how to work with ideas of randomness, probability, expected value, prediction, estimation,
etc., in ways that are sensible and mathematically clear.

There are also other sources of randomness beside uncertainty. For example, computers often use \term{pseudorandom numbers} to make games fun, simulations accurate, and searches efficient. Also, according
to the modern theory of quantum mechanics, the makeup of atomic matter is in some sense \emph{truly} random. All such
sources of randomness can be studied using the techniques of this text.

Another way of thinking about probability is in terms of \term{relative frequency}. For
example, to say a coin has a 50\% chance of coming up heads can be interpreted as saying that, if we flipped coin
many, many, many times, then approximately half of the time it would come up heads. This interpretation has some
limitations. In many cases (such as tomorrow's weather or next week's stock prices), it is impossible to repeat the
experiment many, many, many times. Furthermore, what precisely does ``approximately'' mean in this case? However,
despite these limitations, the relative frequency interpretation is a useful way to think of probabilities and to
develop intuition about them.

Uncertainty has been with us forever, of course, but the mathematical theory of probability originated in the
seventeenth century. In 1654, the Paris gambler Le Chevalier de M\'er\'e asked Blaise Pascal about certain
probabilities that arose in gambling (such as, if a game of chance is interrupted in the middle, what is the
probability that each player would have won had the game continued?). Pascal was intrigued and corresponded with the
great mathematician and lawyer Pierre de Fermat about these questions. Pascal later wrote the book \emph{Trait\'e du
Triangle Arithmetique}, discussing binomial coefficients (Pascal's triangle) and the binomial probability
distribution.

At the beginning of the twentieth century, Russians such as Andrei Andreyevich Markov, Andrey Nikilayevich
Kolmogorov, and Pafnuty L. Chebychev (and American Norbert Wiener) developed a more formal mathematical theory of
probability. In the 1950s, Americans William Feller and Joe Doob wrote important books about the mathematics of
probability theory. They popularized the subject in the western world, both as an important area of pure mathematics
and as having important applications in physics, chemistry, and later in computer science, economics, and finance.

\subsection{Why Do We Need Probability Theory?}
Probability theory comes up very often in our daily lives. We offer a few examples here.

Suppose you are considering buying a ``Lotto 6/49'' lottery ticket. In this lottery, you are to pick six distinct
integers between 1 and 49. Another six distinct integers between 1 and 49 are then selected at random by the lottery
company. If the two sets of six integers are identical, then you win the jackpot.

After mastering \autoref{unif_prob_fint_spc}, you will know how to calculate that the probability of the two sets
matching is equal to one chance in 13,983,816. This is, it is about 14 million times more likely that you will not
win the jackpot than that you will. (These are not very good odds!)

Suppose the lottery ticket cost \$1 each. After mastering expected values in \autoref{chap:expect}, you will know
that you should not even \emph{consider} buying a lottery ticket unless the jackpot is more than \$14 million (which
it usually is not). Furthermore, if the jackpot is ever more than \$14 million, then likely many other people will
buy lottery tickets that week, leading to a larger probability that you will have to \emph{share} the jackpot with
other winners even if you do win --- so it is probably not in your favor to buy a lottery ticket even then.

Suppose instead that a ``friend'' offers you a bet. He has three cards, one red on both sides, one black on both
sides, and one red on one side  and black on the other. He mixes the three cards in a hat, picks one at random, and
places it flat on the table with the only one side showing. Suppose that one side is red. He then offers to bet his
\$4 against your \$3 that the other side of the card is also red.

At first you might think it sounds like the probability that the other side is also red is 50\% thus, a good bet.
However, after mastering conditional probability (\autoref{cond_prob_indp}), you will know that, conditional on one
side being red, the conditional probability that the other side is also red is equal to 2/3. So, by the theory of
expected values (\autoref{chap:expect}), you will know that you should not accept your ``friend's'' bet.

Finally, suppose he suggests that you flip a coin one thousand times. Your ``friend'' says that if the coin comes up
heads at least six hundred times, then  he will  pay you \$100; otherwise, you have to pay him just \$1.

At first you might think that, while 500 heads is the most likely, there is still a \emph{reasonable} chance that 600
heads will appear --- at least good enough to justify accepting your friend's \$100 to \$1 bet. However, after
mastering the laws of large numbers (\autoref{chap:sampl_dist_n_lim}), you will know that as the number of coin flips
get large, it become more and more likely that number of heads is very close to half of the total number of coin
flips. In fact, in this case, there is less than one chance in ten billion of getting more than 600 heads! Therefore,
you should not accept this bet, either.

As these examples show, a good understanding of probability theory will allow you to correctly assess probabilities
in everyday situations, which will in turn allow you to make wiser decisions. It might even save you money!

Probability theory also plays a key role in many important applications of science and technology. For example, the
design of a nuclear reactor must be such that the escape of radioactivity into the environment is an extremely rare
event. Of course, we would like to say that it is categorically impossible for this to ever happen, but reactors are
complicated systems, built up from many interconnected subsystems, each of which we know will fail to function
properly at some time. Furthermore, we ca never definitely say that a natural event like an earthquake cannot occur
that would damage the reactor sufficiently to allow an emission. The best we can do is try to quantify our
uncertainty concerning the failures of reactor components or the occurrence of natural events that would lead to such
an event.  This is where probability enters the picture. Using probability as a tool to deal with the uncertainties,
the reactor can be designed to ensure that an unacceptable emission has an extremely small probability --- say once
in a billion years -- of occurring.

The gambling and nuclear reactor examples deal essentially with the concept of \term{risk} --- the risk of losing
money, the risk of being exposed to at injurious level of  radioactivity, etc. In fact, we are exposed to risk all the
 time. When we ride in a car, or take an airplane flight, or even walk down the street, we are exposed to risk. We
 know that the risk of injury in such circumstances is never zero, yet we still engage in these activities. This is
 because we intuitively realize that the probability of an accident occurring is extremely low.

 So we are using probability every day in our lives to assess risk. As the problems we face, individually or
 collectively, become more complicated, we need to refine and develop our rough, intuitive ideas about probability to
 form a clear and precise approach. This is why probability theory has been developed as a subject. In fact, the
 insurance industry has been developed to help us cope with risk. Probability is the tool used to determine what you
 pay to reduce your risk or to compensate you or your family in case of a personal injury.

\begin{summary}
    \item Probability theory provides us with a precise understanding of uncertainty.
    \item This understanding can help us make predictions, make better decisions, assess risk, and even make money.
 %   \item А по-русски ты умеешь ли ? Давай  вот сейчас и проверим, да или нет. Метрополитен Подземелье Черезмерный.
\end{summary}

\begin{discussion}
    \item Do you think that tomorrow's weather and next week's stock prices are ``really'' random, or is just a
    convenient way to discuss and analyze them?
    \item Do you think it is possible for probabilities to depend on who is observing them, or at what time?
    \item Do you find it surprising that probability theory was not discussed as a mathematical subject until the
    seventeenth century? Why or why not?
    \item In what ways is probability important for such subjects as physics, computer science, finance? Explain.
    \item What are examples from your own life where thinking about probabilities did save --- or could have saved
    --- you money or helped you to make a better decision? (List as many as you can.)
    \item Probabilities are often depicted in popular movies and television programs. List as many examples as you
    can. Do you think the probabilities were portrayed there in a ``reasonable'' way?
\end{discussion}
\section{Probability Models}
A formal definition of probability begins with a \term{sample space}, often written  $S$. This
sample space is any set that lists all possible \term{outcomes} (or, \term{responses})
of some unknown experiment or situation. For example, perhaps
\begin{equation}
    S =  \text{\{rain, snow, clear\}} \notag
\end{equation}
when predicting tomorrow's weather. Or perhaps $S$ is the set of all positive real numbers, when predicting next
week's stock price. The point is, $S$ can be any set at all, even an infinite. We usually write $s$ for an element of
$S$, so that $s \in S$. Note that $S$ describes only those things that we are interested in; if we are studying
weather, then rain and snow are in $S$, but tomorrow's stock prices are not.

A probability model also requires a collection of \term{events}, which are subsets of $S$ to which
probabilities can be assigned. For the above weather example, the subsets \{rain\}, \{snow\}, \{rain,snow\}, \{rain,
clear\}, \{rain, snow, clear\}, and even the empty set $\emptyset = \{\,\}$ are all examples of subsets of $S$ that
could be events. Note that here the comma means ``or''; thus, \{rain, snow\} is the event that it will rain \emph{or}
snow. We will generally assume that \emph{all} subsets of $S$ are events. (In fact, in complicated situations there
are some technical restrictions on what subsets can or cannot be events, according to the mathematical subject of
measure theory. But we will not concern ourselves with such technicalities here.)

Finally, and most importantly, a probability model requires a \term{probability measure},
usually written $P$. This probability measure must assign, to each event $A$, a probability $P(A)$. We require the
following properties:
\begin{enumerate}
    \item $P(A)$ is always a nonnegative real number, between 0 and 1 inclusive.
    \item $P(\emptyset) = 0$, i.e., if $A$ is the empty set $\emptyset$, then $P(A)=0$.
    \item $P(S)=1$, i.e., if $A$ is the entire sample space $S$, then $P(A)=1$.
    \item $P$ is \emph{(countably) additive}, meaning that if $A_1, A_2, \dots$ is finite or countable sequence of
    disjoint events, then
    \begin{equation}\label{eq:additive_rule}
        P(A_1 \cup A_2 \cup \ldots) = P(A_1) + P(A_2) + \ldots\,.
    \end{equation}
\end{enumerate}

The first of these properties says that we shall measure all probabilities on a scale from 0 to 1, where 0 means
impossible and 1 (or 100\%) means certain. The second property says the probability that \emph{nothing} happens is 0;
in other words, it is impossible that \emph{no} outcome will occur. The third property says the probability that
\emph{something} happens is 1; in other words, it is certain that \emph{some} outcome must occur.

The fourth property is the most subtle. It says that we can calculate probabilities of complicated events by adding
up  the probabilities of smaller events, provided those smaller events are \emph{disjoint} and together contain the
entire complicated event. Note that events  are \term{disjoint} if they contain no outcomes in
common. For example, \{rain\} and \{snow, clear\} are disjoint, whereas \{rain\} and \{rain, clear\} are not
disjoint. (We are assuming for simplicity that it cannot both rain \emph{and} snow tomorrow). Thus, we should have
$P(\{\text{rain}\}) + P(\{\text{snow, clear}\}) = P(\{\text{rain, snow, clear}\})$, but we do \emph{not} expect to
have $P(\{\text{rain}\}) + P(\{\text{rain, clear}\}) = P(\{\text{rain, rain, clear}\})$ (the latter being the same as
$P(\{\text{rain, clear}\})$).

We now formalize the definition of a probability model.
\begin{definition}[Probability model]
    A \term{probability model}    consists of a nonempty set called the sample space $S$; a collection of events that
    are subsets of $S$; and a probability measure $P$ assigning a probability between 0 and 1 to each event, with
    $P(\emptyset)=0$ and $P(S)=1$ and with $P$ additive as in \eqref{eq:additive_rule}.
\end{definition}

\begin{example}
    Consider again the weather example, with $S = \{\text{rain, snow, clear}\}$. Suppose that the probability of rain
    is 40\%, the probability of snow is 15\%, and the probability of a clear day is 45\%. We can express this as
    $P(\{\text{rain}\})=0.40$, $P(\{\text{snow}\})=0.15$, and $P(\{\text{clear}\})=0.45$.

    For this example, of course $P(\emptyset)=0$, i.e., it is impossible that \emph{nothing} will happen tomorrow.
    Also $P(\{\text{rain, snow, clear}\})=1$, because we are assuming that exactly \emph{one} of rain, snow, or clear
    must occur tomorrow. (To be more realistic, we might say that we are predicting the weather at exactly 11:00 A.M.
    tomorrow.) Now, what is the probability that it will rain \emph{or} snow tomorrow? Well, by additivity property,
    we see that
    $$
    P(\{\text{rain, snow}\}) = P(\{\text{rain}\}) + P(\{\text{snow}\}) = 0.40 + 0.15 = 0.55\,.
    $$
    We thus conclude that, as expected, there is a 55\% chance of rain \emph{or} snow tomorrow.
\end{example}

\begin{example}
    Suppose your candidate has a 60\% chance of winning an election in progress.Then $S=\{\text{win,lose}\}$, with
    $P(\text{win})=0.6$ and $P(\text{lose})=0.4$\,. Note that $P(\text{win})+P(\text{lose})=1$\,.
\end{example}

\begin{example}
    Suppose we flip a fair coin, which can come up either heads ($H$) or tails ($T$) with equal probability. Then
    $S=\{H, T\}$, with $P(H)=P(T)=0.5$\,. Of course, $P(H)+P(T)=1$\,.
\end{example}

\begin{example}
    Suppose we flip three fair coins in a row and keep track of the sequence of heads and tails that result. Then
    $$
    S = \{HHH, HHT, HTH, HTT, THH, THT, TTH, TTT\}.
    $$
    Furthermore, each of these eight outcomes is equally likely. Thus, $P(HHH)=1/8$, $P(TTT)=1/8$, etc. Also, the
    probability that the first coin is heads \emph{and} the second coin is tails, but the third coin can be anything,
    is equal to the sum of the probabilities of the event $HTH$ and $HTT$, i.e., $P(HTH)+P(HTT)=1/8+1/8=1/4$\,.
\end{example}

\begin{example}
    Suppose we flip three fair coins in a row but care only about the number of heads that result. Then
    $S=\{0,1,2,3\}$\,. However, the probabilities of these four outcomes are \emph{not} equally likely; we will see
    later that in fact $P(0)=P(3)=1/8$, while $P(1)=P(2)=3/8$\,.
\end{example}


We note that it is possible to define probability models on more complicated (e.g., uncountably infinite) sample
spaces as well.

\begin{example}
    Suppose that $S=[0,1]$ is the unit interval. We can define a probability measure $P$ on $S$ by saying that
    \begin{equation}
        P([a,b]) = b-a,\qquad \text{whenever $0 \leqslant a \leqslant b \leqslant 1$\,.}
    \end{equation}

    In words, for any\footnote{For the uniform distribution on $[0,1]$, it turns out that not all subsets of $[0,1]$
    can properly be regarded as \emph{events} for this model. However, this is merely a technical property, and any
    subset that we can explicitly write down will always be an event. See more advanced probability books,
    e.g.,~\cite[page 3]{B1}} subinterval $[a,b]$ of $[0,1]$, the probability of the interval is simply the
    \emph{length} of the interval. This example is called the \term{uniform distribution on $[0,1]$}. The uniform
    distribution is just the first of many distributions on uncountable state spaces. Many
    further examples will be given in \autoref{chap:rvar_and_dist}.
\end{example}

\subsection{Venn Diagrams and Subsets}
\term{Venn diagrams} provide a very useful graphical method for depicting the sample space $S$ and subsets of it. For
example, in \ref{venn_1} we have a Venn diagram showing the subset $A \subset S$ and the \term{complement}
$$
A^c = \{s:s\notin A\}
$$
of $A$. The rectangle denotes the entire sample space $S$. The circle (and its interior) denotes the subset~$A$; the
region outside the circle, but inside $S$ denotes $A^c$.
\begin{figure}[hb]
\missingfigure{Venn 1}
\caption{Venn diagram of the subsets $A$ and $A^c$ of the sample space $S$.}\label{venn_1}
\end{figure}

Two subsets $A \subset S$ and $B \subset S$ are depicted as two circles, as in \autoref{venn_2}. The
\term{intersection}
$$
A \cap B = \{s:s \in A \text{ and } s \in B\}
$$
of subsets $A$ and $B$ is the set of elements  common to both sets and is depicted by the region where the two
circles overlap. The set
$$
A \setminus B^c = \{s:s\in A \text{ and } s \notin B\}
$$
is called the \term{complement} \emph{of $B$ in $A$} and is depicted as the region inside the $A$ circle, but not
inside the $B$ circle. This is the set of elements in   $A$ but not in $B$. Similarly, we have the complement of $A$
in $B$, namely, $A^c \cap B$. Observe that the sets $A\cap B$, $A \cap B^c$, and $A^c \cap B$ are mutually disjoint.

The \term{union}
$$
A \cup B = \{ s : s \in A \text{or} s \in B\}
$$
of sets $A$ and $B$ is the set of elements that are in either $A$ or $B$. In \autoref{venn_2}, it is depicted by the
region covered by both circles. Notice that $ A \cup B = \left(A \cap B^c\right) \cup (A \cap B) \cup (A^c \cap B)$\,.


There is one further region in \autoref{venn_2}. This is the complement of $A \cup B$, namely, the set of elements
that
are in neither $A$ nor $B$. So we immediately have
$$
    (A\cup B)^c = A^c \cap B^c\,.
$$
Similarly, we can show that
$$
    (A\cap B)^c = A^c \cup B^c
    \,,
$$
namely, the subset of elements that are not in both $A$ and $B$ is given by the set of elements not in $A$ or not in
$B$.
\begin{figure}[hb]
    \missingfigure{Venn 2}
    \caption{Venn diagram depicting the subsets $A$, $B$, $A\cap B$, $A\cap B^c$, $A^c \cap B$, $A^c \cap B^c$ and $A
    \cup B$.}\label{venn_2}
\end{figure}


Finally, we note that if $A$ and $B$ are disjoint subset, then it makes sense to depict these as drawn in
\autoref{venn_3}, i.e., as two nonoverlapping circles because they have no elements in common.
\begin{figure}[hb]
    \missingfigure{Venn 3}
    \caption{Venn diagram of disjoint subsets $A$ and $B$.}\label{venn_3}
\end{figure}

\begin{summary}
    \item A probability model consists of a sample space $S$ and a probability measure $P$ assigning probabilities to
    each event.
    \item Different sorts of sets can arise as sample spaces.
    \item Venn diagrams provide a convenient method for representing sets and the relationships among them.
\end{summary}

\begin{exercises}
    \item Suppose $S=\{1,2,3\}$, with $P(\{1\})=1/2$, $P(\{2\})=1/3$, and $P(\{3\})=1/6$.
        \begin{enumerate}
            \item What is $P(\{1,2\})$?
            \item What is $P(\{1,2,3\})$?
            \item List all events $A$ such that $P(A)=1/2$.
        \end{enumerate}
    \item Suppose $S=\{1,2,3,4,5,6,7,8\}$, with $P(\{s\})=1/8$ for $1 \leqslant s \leqslant 8$.
        \begin{enumerate}
            \item What is $P(\{1,2\})$?
            \item What is $P(\{1,2,3\})$?
            \item How many events $A$ are there such that $P(A) = 1/2$?
        \end{enumerate}
    \item Suppose $S=\{1,2,3\}$, with $P(\{1\})=1/2$ and $P(\{1,2\})=2/3$. What must $P(\{2\})$ be?
    \item Suppose $S=\{1,2,3\}$, and we try to define $P$ by $P(\{1,2,3\})=1$, $P(\{1,2\})=0.7$, $P(\{1,3\})=0.5$,
    $P(\{2,3\})=0.7$, $P({1})=0.2$, $P(\{2\})=0.5$, $P(\{3\})=0.3$. Is $P$ a valid probability measure? Why or why
    not?
    \item Consider the uniform distribution on $[0,1]$. Let $s \in [0,1]$ be any outcome. What is $P(\{s\})$? Do you
    find this result surprising?
    \item Label the subregions in the Venn diagram in \autoref{venn_4} using the sets $A$, $B$, and $C$ and their
    complements (just as we did in \autoref{venn_2}).
    \item On Venn diagram, depict the set of elements that are in subsets $A$ or $B$ but \emph{not} in both. Also
    write this as a subset involving unions and intersections of $A$, $B$, and their complements.
    \item Suppose $S=\{1,2,3\}$, and $P(\{1,2\})=1/3$, and $P(\{2,3\})=2/3$. Compute $P(\{1\})$, $P(\{2\})$, and
    $P(\{3\})$.
    \item Suppose $S=\{1,2,3,4\}$, and $P(\{1\})=1/12$, and $P(\{1,2\})=1/6$, and $P(\{1,2,3\})=1/3$. Compute
    $P(\{1\})$, $P(\{2\})$, $P(\{3\})$, and $P(\{4\})$.
    \item Suppose $S=\{1,2,3\}$, and $P(\{1\})=P(\{3\})=2P(\{2\})$. Compute $P(\{1\})$, $P(\{2\})$, and $P(\{3\})$.
    \item Suppose $S=\{1,2,3\}$, and $P(\{1\})=P(\{2\})+1/6$, and $P(\{3\})=2P(\{2\})$. Compute $P(\{1\})$,
    $P(\{2\})$, and $P(\{3\})$.
    \item Suppose $S=\{1,2,3,4\}$, and $P(\{1\})-1/8=P(\{2\})=3P(\{3\})=4P(\{4\})$. Compute $P(\{1\})$, $P(\{2\})$,
    $P(\{3\})$ and $P(\{4\})$.
\end{exercises}
\begin{figure}[h]
    \missingfigure{Venn 4}
    \caption{Venn diagram of subsets $A$, $B$ and $C$.}\label{venn_4}
\end{figure}

\begin{problems}
    \item Consider again the uniform distribution on $[0,1]$. Is it true that
    $$
    P([0,1]) = \sum_{s \in[0,1]}P(\{s\})\quad\text{?}
    $$
    How does this relate to the additivity property of probability measures?
    \item Suppose $S$ is a finite or countable set. Is it possible that $P(\{s\})=0$ for every single $s\in{}S$? Why
    or why not?
    \item Suppose $S$ is an uncountable set. Is it possible that $P(\{s\})=0$ for every single $s\in S$? Why or why
    not?
\end{problems}

\begin{discussion}
    \item Does the additivity property make sense intuitively? Why or why not?
    \item Is it important that we always have $P(S)=1$\,? How would probability theory change if this were not the
    case?
\end{discussion}
\section{Properties of Probability Models}
The additivity property of probability measure automatically implies certain basic properties. These are true for
\emph{any} probability model at all.

If $A$ is any event, we write $A^c$ (read ``A complement'') for the event that $A$ does \emph{not} occur. In the
weather example, if $A=\{\text{rain}\}$, then $A^c=\{\text{snow, clear}\}$. In the coin examples, if $A$ is the event
that the first coin is heads, then $A^c$ is the event that first coin is tails.

Now, $A$ and $A^c$ are always disjoint. Furthermore, their union is always the entire sample space: $A \cup A^c = S$.
Hence, by the additivity property, we must have $P(A) + P(A^c) = P(S)$. But we always have $P(S)=1$. Thus, $P(A) +
P(A^c)=1$, or
\begin{equation}
    \label{eq:131}
    P(A^c)=1 - P(A)\,.
\end{equation}
In words, the probability that any event does \emph{not} occur is equal to one minus the probability that it
\emph{does} occur. This is a very helpful fact that we shall use often.

Now suppose that $A_1, A_2, \ldots$ are events that form a \term{partition} of the sample space $S$. This means that
$A_1, A_2, \ldots$ are disjoint and, furthermore, that their union is equal to $S$, i.e., $A_1 \cup A_2 \cup \ldots =
S$. We have the following basic theorem that allows us to decompose the calculation of the probability of $B$ into
the sum of the probabilities of the sets $A_i \cap B$. Often these are easier to compute.

\begin{theorem}[\term{Law of total probability}, \emph{unconditioned version}]
    Let $A_1, A_2, \ldots$ be events that form a partition of the sample space $S$. Let $B$ be any event. Then
    $$
    P(B) = P(A_1 \cap B) + P(A_2 \cap  B) + \ldots\,.
    $$
\end{theorem}
\begin{proof}
    The events $(A_1 \cap B)$, $(A_2\cap B)$, $\ldots$ are disjoint, and their union is $B$. Hence, the result
    follows immediately from the additivity property \autoref{eq:additive_rule}.
\end{proof}
A somewhat more useful version of the law of total probability, and applications of its use, are provided in
\autoref{cond_prob_indp}.

Suppose now that $A$ and $B$ are two events such that $A$ \emph{contains} $B$ (in symbols, $A \supseteq B$).
In words, all outcomes in $B$ are also in $A$. Intuitively, $A$ is ``large'' event than $B$, so we would expect its
probability to be larger.
We have the following result.
\begin{theorem}
    Let $A$ and $B$ be two events with $A \supseteq B$. Then
    \begin{equation}
        \label{eq:132}
        P(A) = P(B)+P(A \cap B^c)\,.
    \end{equation}
\end{theorem}
\begin{proof}
    We can write $A=B\cup (A \cap B^c)$, where $B$ and $A \cap B^c$ are disjoint. Hence, $P(A)=P(B)+P(A\cap B^c)$ by
    additivity.
\end{proof}
Because we always have $P(A \cap B^c)\geqslant 0$, we conclude the following.
\begin{corollary}[\term{Monotonicity}]
    Let $A$ and $B$ be two events, with $ A \supseteq B$. Then
    $$
    P(A) \leqslant P(B)\,.
    $$
\end{corollary}
On the other hand, rearranging \autoref{eq:132}, we obtain the following.
\begin{corollary}\label{cor:132}
    Let $A$ and $B$ be two events, with $A \supseteq B$. Then
    \begin{equation}
        P(A \cap B^c) = P(A) - P(B)\,.
    \end{equation}
\end{corollary}

More general, even if we do not have $A \supseteq B$, we have the following property.

\begin{theorem}[\term{Principle of inclusion-exclusion}, \emph{two-event version}]
    Let $A$ and $B$ be two events. Then
    \begin{equation}
        P(A \cup B) = P(A) + P(B) - P(A \cap B)\,.
    \end{equation}
\end{theorem}
\begin{proof}
    We can write $A\cup B = (A \cap B^c) \cup (B \cap A^c) \cup (A \cap B)$, where  $A\cap B^c$, $B \cap A^c$, and
    $A\cap B$ are disjoint. By additivity we have
    \begin{equation}\label{eq:prf_1}
        P(A\cup B) = P(A\cap B^c) + P(B \cap A^c)+P(A \cap B)\,.
    \end{equation}
    On the other hand, using \autoref{cor:132}(with $B$ replaced by $A \cap B$), we have
    \begin{equation}\label{eq:prf_2}
        P(A \cap B^c) = P(A \cap (A \cap B)^c) = P(A) - P(A \cap B)
    \end{equation}
    and similarly,
    \begin{equation}\label{eq:prf_3}
        P(B \cap A^c) = P(B) - P(A \cap B)\,.
    \end{equation}
    Substituting \autoref{eq:prf_2} and \autoref{eq:prf_3} into \autoref{eq:prf_1}, the result follows.
\end{proof}
A more general version of the principle of inclusion-exclusion is developed in \autoref{challenge:1310}.

Sometimes we do not need to evaluate the probability content of a union; we need only know it is bounded above by sum
of the probabilities of the individual events. This is called subadditivity.
\begin{theorem}[\term{Subadditivity}]
    Let $A_1, A_2,\ldots$ be a finite or countably infinite sequence of events, not necessarily disjoint. Then
    $$
    P(A_1\cup A_2\cup\ldots)\leqslant P(A_1)+P(A_2)+\ldots
    \,.
    $$
\end{theorem}
\begin{proof}
    See \autoref{ch2:adv_proofs} for the proof of this result.
\end{proof}

We note that some properties in the definition of a probability model actually follow from other properties. For
example, once we know the probability $P$ is additive and that $P(S)=1$, it follows that we \emph{must} have
$P(\emptyset)=0$. Indeed, because $S$ and $\empty$ are disjoint, $P(S\cup \emptyset)=P(S)+P(\emptyset)$. But of
course, $P(S \cup \emptyset) = P(S)=1$, so we must have $P(\emptyset)=0$.

Similarly, once we know $P$ is additive on countably infinite sequences of disjoint events, it follows that $P$ must
be additive on finite sequences of disjoint events, too. Indeed, given a finite disjoint sequence $A_1, \ldots, A_n$,
we can just set $A_i=\emptyset$ for all $i > n$, to get a countably infinite disjoint sequence with same union and
the same sum of probabilities.

\begin{summary}
    \item The probability of the complement of an event equals one minus the probability of the event.
    \item Probabilities always satisfy the basic properties of total probability, subadditivity, and monotonicity.
    \item The principle of inclusion-exclusion allows for the computation of $P(A \cup B)$ in terms of simpler events.
\end{summary}

\begin{exercises}
    \item Suppose $S=\{1,2,\ldots,100\}$. Suppose further that $P(\{1\})0.1$.
    \begin{enumerate}
        \item What is the probability $P(\{2,3,\ldots,100\})$?
        \item What is the smallest possible value of $P(\{1,2,3\})$?
    \end{enumerate}
    \item Suppose that Al watches the six o'clock news 2/3 of the time, watches the eleven o'clock news 1/2 of the
    time, and watches both the six o'clock and eleven o'clock news 1/3 of the time. For a randomly selected day, what
    is the probability that Al watches neither news?
    \item Suppose that an employee arrives late 10\% of the time, leaves early 20\% of the time, and both arrives
    late \emph{and} leaves early 5\% of the time. What is the probability that on a given day that employee will
    either arrive late \emph{or}  leave early (or both) ?
    \item Suppose your right knee is sore 15\% of the time, and your left knee is sore 10\% of the time. What is the
    largest possible percentage of time that at least one of your knees is sore? What is the smallest possible
    percentage of time that at least one of your knees is sore?
    \item Suppose a fair coin is flipped five times in a row.
    \begin{enumerate}
        \item What is the probability of getting all five heads?
        \item What is the probability of getting at least one tail?
    \end{enumerate}
    \item Suppose a card is chosen uniformly at random from a standard 52-card deck.
    \begin{enumerate}
        \item What is the probability that the card is a jack?
        \item What is the probability that the card is a club?
        \item What is the probability that the card is both a jack and a club?
        \item What is the probability that the card is either a jack or a club (or both)?
    \end{enumerate}
    \item Suppose your team has a 40\% chance of winning or tying today's game and has a 30\% chance of winning
    today's game. What is the probability that today's game will be a tie?
    \item Suppose 55\% of students are female, of which 4/5(44\%) have long hair, and 45\% are male, of which 1/3
    (15\% of all students) have long hair. What is the probability that a student chosen at random will either be
    female or have long hair (or both)?
\end{exercises}

\begin{problems}
    \item Suppose we choose a positive integer at random, according to some unknown probability distribution. Suppose
    we know that $P(\{1,2,3,4,5\})=0.3$, that $P({4,5,6})=0.4$, and that $P(\{1\})=0.1$. What are the largest and
    smallest possible values of $P(\{2\})$?
\end{problems}

\begin{challeges}
    \item     \label{challenge:1310} Generalize the principle of inclusion-exclusion, as follows.
    \begin{enumerate}
        \item Suppose there are three events $A$, $B$, and $C$. Prove that
        \begin{equation*}
            P(A \cup B \cup C) = P(A) + P(B) + P(C) - P(A \cap B) - P(A \cap C) - P(B\cap C)+ P(A \cap B \cap
            C)\,.
        \end{equation*}
        \item Suppose there are $n$ events $A_1$, $A_2$, $\ldots$ ,$A_n$. Prove that
        \begin{equation*}
            \begin{split}
                P(A_1 \cup \cdots \cup A_n) &= \sum_{i=1}^{n}P(A_i) - \sum_{\substack{i,j=1 \\ i<j}}^{n}P(A_i
                \cap
                A_j) +
                \sum_{\substack{i,j,k=1 \\ i < j < k}}^{n}P(A_i\cap A_j \cap A_k) \\
                &- \ldots \pm P(A_1 \cap \cdots \cap A_n)\,.
            \end{split}
        \end{equation*}
    \end{enumerate}
    (Hint: Use induction.)
\end{challeges}

\begin{discussion}
    \item Of the various theorems presented in this section, which ones do you think are the most important? Which
    ones do you think are the least important? Explain the reasons for your choices.
\end{discussion}

\section{Uniform Probability on Finite Spaces}\label{unif_prob_fint_spc}
If the sample space $S$ is finite, then one possible probability measure on $S$ is the \term{uniform} probability
measure, which assigns probability $1/|S|$ to each outcome. Here $|S|$ is the number of elements in the sample space
$S$. By additivity, it then follows that for any event $A$ we have
\begin{equation}
    \label{eq:141}
    P(A) = \frac{|A|}{|S|}\,.
\end{equation}
\begin{example}
    Suppose we roll a six-sided die. The possible outcome are $S=\{1,2,3,4,5,6\}$, so that $|S|=6$. If the die is
    fair, then we believe each outcome is equally likely. We thus set $P(\{i\})=1/6$ for each $i \in S$ so that
    $P(\{3\})=1/6$, $P(\{4\})=1/6$, etc. It follows from \autoref{eq:141} that, for example, $P(\{3,4\})=2/6=1/3$,
    $P(\{1,5,6\})=3/6=1/2$, etc. This is a good model of rolling a fair six-sided die once.
\end{example}
\begin{example}
    For a second example, suppose we flip a fair coin once. Then $S=\{\text{heads, tails}\}$, so that $|S|=2$, and
    $P(\{\text{heads}\})=P(\{\text{tails}\})=1/2$.
\end{example}
\begin{example}
    Suppose now that we flip \emph{three different} fair coins. The outcome can be written as a sequence of three
    letters, with each letter being $H$ (for heads) of $T$ (for tails). Thus,
    $$
    S = \{HHH, HHT, HTH, HTT, THH, THT, TTH, TTT\}\,.
    $$
    Here $|S|=8$, and each of the events is equally likely. Hence, $P(\{HHH\})=1/8$, $P(\{HHH,TTT\})=2/8=1/4$, etc.
    Note also that, by additivity, we have, for example, that $P(\text{exactly two heads}) = P(\{HHT, HTH,
    THH\})=1/8+1/8+1/8=3/8$, etc.
\end{example}
\begin{example}
    For a final example, suppose we roll a fair six-sided die \emph{and} flip a fair coin. Then we can write
    $$
    S=\{1H,2H,3H,4H,5H,6H,1T,2T, 3T, 4T, 5T, 6T\}\,.
    $$
    Hence, $|S|=12$ in this case, and $\forall s \in S, P(s)=1/12$.
\end{example}

\subsection{Combinatorial Principles}
Because of \autoref{eq:141}, problems involving uniform distributions on finite sample spaces often come down to
being able to compute the sizes $|A|$ and $|S|$ of the sets involved. That is, we need to be good at \emph{counting}
the number of elements in various sets. The science of counting is called \term{combinatorics}, and some aspects of
it are very sophisticated. In the remainder of this section, we consider a few simple combinatorial rules and their
application in probability theory when the uniform distribution is appropriate.

\begin{example}[Counting Sequences: The Multiplication Principle]
    Suppose we flip three fair coins and roll two fair six-sided dice. What is the probability that all three coins
    come up heads and that both dice come up 6? Each coin has two possible outcomes (heads or tails), and each die
    has six  possible outcomes $\{1,2,3,4,5,6\}$. The total number of possible outcomes of the three coins and two
    dice is thus given by \emph{multiplying} three 2's and two 6's, i.e., $2\times2\times2\times6\times6=288$. This is
    sometimes referred as the \term{multiplication principle}. There are thus 288 possible outcomes of our experiment
    (e.g., $HHH66$, $HTH24$, $TTH15$, etc.). Of these outcomes, only one (namely, $HHH66$) counts as a success. Thus,
    the probability that all three coins come up heads and both dice come up 6 is equal to 1/288.

    Notice that we can obtain this result in an alternative way. The chance that any one of the coins comes up heads
    is 1/2, and the chance that any one die comes 6 is 1/6. Furthermore, these events are all \emph{independent} (see
    next section). Under independence, the probability that they \emph{all} occur is given by the product of their
    individual probabilities, namely,
    $$
    (1/2)\times(1/2)\times(1/2)\times(1/6)\times(1/6)=1/288\,.
    $$

    More generally, suppose we have $k$ finite sets $S_1,\dots,S_k$ and we want to count the number of sequences of
    length $k$ where the $i$th element comes from $S_i$, i.e., count the number of elements in
    $$
    S = \{(s_1,\dots,s_k) : s \in S_i\} = S_1 \times \dots \times S_k\,.
    $$
    The multiplication principle says that the number of such sequences is obtained by multiplying together the
    number of elements in each set $S_i$, i.e.,
    $$
        |S| = |S_1|\times\dots\times|S_k|\,.
    $$
\end{example}

\begin{example}
    Suppose we roll two fair six-sided dice. What is the probability that the sum of the numbers showing is equal to 10? By the above multiplication principle, the total number of possible outcomes is equal to $6\times6=36$. Of these outcomes, there are three that sum to 10, namely, (4,6), (5,5), (6,4). Thus, the probability that the sum is 10 is equal to 3/36 or 1/12.
\end{example}

\begin{example}[\term{Counting permutations}]
    Suppose four friends go to a restaurant, and each checks his or her coat. At the end of the meal, the four coats are \emph{randomly} returned to the four people. What is the probability that each of the four people gets his or her own coat? Here the total number of different ways the coats can be returned  is equal to $4\times3\times2\times1$, or $4!$ (i.e., four factorial). This is because the first coat can be returned to any of the four friends, the second coat to any of the three remaining friends, and so on. Only one of these assignments is correct. Hence, the probability that each if the four people gets his or her own coat is equal to $1/4!$, or 1/24.

    Here we are counting permutations, or sequences of elements from a set where no element appears more than once. We can use the multiplication principle to count permutations more generally. For example, suppose $|S|=n$ and we want to count the number of permutations of length $k \leqslant n$ obtained from $S$, i.e., we want to count the number of elements of the set
   $$
   \{(s_1,\dots,s_k):s\in S, s_i \neq s_j \text{ when } i \neq j\}\,.
   $$
   Then we have $n$ choices for the first element $s_1$, $n-1$ choices for the second element, and finally $n-(k-1)=n-k+1$ choices for the last element. So there are $n\times(n-1)\times\dots\times(n-k+1)$ permutations of length $k$ from a set of $n$ elements. This can also be written as $n! / (n-k)!$. Notice that when $k=n$, there are
   $$
   n!=n\times(n-1)\times\dots\times2\times1
   $$
   permutations of length $n$.
\end{example}

\begin{example}[Counting Subsets]
    Suppose 10 fair coins are flipped. What is the probability that exactly seven of them are heads? Here each possible sequence of 10 heads or tails (e.g., $HHHTTTHTTT$, $THTTTTHHHT$, etc.) is equally likely, and by the multiplication principle the total number of possible outcomes is equal to 2 multiplied by itself 10 times, or $2^{10}=1024$. But of these sequences, how many have exactly seven heads?

    To answer this, notice that we may specify such a sequence by giving the positions of the seven heads, which involves choosing a subset of size 7 from the set of possible indices $\{1,\dots,10\}$. There are $10!/3!=10\times9\times\dots\times5\times4$ different permutations of length 7 from $\{1,\dots,10\}$, and each such permutation specifies a sequence of seven heads and three tails. But we can permute the indices specifying where the heads go in $7!$ different ways without changing the sequence of heads and tails. So the total number of outcomes with exactly seven heads is equal to $10!/(3!\cdot7!)=120$. The probability that exactly seven of the 10 coins are heads is therefore equal to $120/1024$, or just under 12\%.

    In general, if we have a set $S$ of $n$ elements, then the number of different subsets of size $k$ that we can construct by choosing elements from $S$ is
    $$
        \binom n k = \frac{n!}{k!(n-k)!}\,,
    $$
    which is called the \term{binomial coefficient}. This follows by the same argument, namely, there are $n!/(n-k)!$ permutations of length $k$ obtained from the set; each such permutation, and $k!$ permutations obtained by permuting it, specify a unique subset of $S$.
\end{example}

It follows, for example, that the probability of obtaining exactly $k$ heads when flipping a total of $n$  fair coins is  given by
$$
    \binom{n}{k}2^{-n} = \frac{n!}{k!(n-k)!}2^{-n}\,.
$$
This is because there are $\binom n k$ different patterns of $k$ heads and $n-k$ tails, and a total of $2^n$ different sequences of $n$  heads and tails.

More generally, if each coin has probability $\theta$ of being heads (and probability $1-\theta$ of being tails), where $0 \leqslant \theta \leqslant 1$, then the probability of obtaining exactly $k$ heads when flipping a total of $n$ such coins is given by
\begin{equation}
    \binom{n}{k}\cdot\theta^k\cdot(1-\theta)^{n-k} = \frac{n!}{k!(n-k)!}\cdot\theta^k\cdot(1-\theta)^{n-k}\,,
\end{equation}
because each of the $\binom{n}{k}$ different patterns of $k$ heads and $n-k$ tails has probability $\theta^k(1-\theta)^{n-k}$ of occurring (this follows from the discussion of independence in \autoref{independence_of_events}). If $\theta=1/2$, then this reduces to the previous formula.

\begin{example}
    \todo{add it}
\end{example}

\begin{summary}
    \item \todo{add it}
    \item \todo{add it}
\end{summary}

\begin{exercises}
    \item \todo{add it}
    \item \todo{add it}
    \item \todo{add it}
    \item \todo{add it}
    \item \todo{add it}
    \item \todo{add it}
    \item \todo{add it}
    \item \todo{add it}
    \item \todo{add it}
    \item \todo{add it}
    \item \todo{add it}
    \item \todo{add it}
    \item \todo{add it}
\end{exercises}

\begin{problems}
    \item \todo{add it}
    \item \todo{add it}
    \item \todo{add it}
    \item \todo{add it}
    \item \todo{add it}
    \item \todo{add it}
\end{problems}

\begin{challeges}
    \item \todo{add it}
    \item \todo{add it}
\end{challeges}

\section{Conditional Probability and Independence}\label{cond_prob_indp}
\lipsum[1-10]

\subsection{Conditional Probability}
\lipsum[1-10]

\subsection{Independence of Events}\label{independence_of_events}
\lipsum[1-10]

\section{Continuity of $P$}
\lipsum[1-10]

\section{Further Proofs (Advanced)}\label{ch2:adv_proofs}
\lipsum[1-10]